% !TeX root = ../thuthesis-example.tex

\chapter{固体火箭发动机课程设计}

装药是指填入燃烧室的推进剂药柱,它的形状和尺寸对发动机的性能至关重要。装药的几何特征直接影响燃气的生成速率和变化规律,从而决定了发动机推力和压力随时间的变化。

\section{课程设计的内容}

\begin{enumerate}[leftmargin=2em]
    \item 装药药形选择
    \item 按给定的原始参数计算装药尺寸(多组)
    \item 给出燃面变化规律(通过曲线给出,肉厚步长取1mm)
    \item 计算对应装药无侵蚀燃烧时的发动机工作$p_{t}$曲线
    \item 计算对应装药有侵蚀燃烧时的发动机工作$p_{t}$曲线
    \item 分析侵蚀燃烧对内弹道的影响
    \item 通过内弹道分析和装药性能分析确定一种装药
\end{enumerate}
\section{固体火箭发动机设计详细参数要求}

\subsection{装药几何参数}

\begin{table}
  \centering
  \caption{装药几何参数}
  \begin{tabular}{ll}
    \toprule
    相关几何参数          & 要求                         \\
    \midrule
    外径   & 230~300mm \\
    长度   & 1300~1800mm                     \\
    总冲   & 不小于230kN·s    \\
    最大推力   & 38~72kN    \\
    平均推力   & 35~65kN    \\
    最小推力   & 31~58kN    \\
    平均工作时间   & 4~7s    \\
    工作环境温度   & 20℃    \\

    \bottomrule
  \end{tabular}
  \label{tab:three-line}
\end{table}

\subsection{进剂种类选择}

在附录A表$1\sim3$中选取。

\subsection{壳体材料选择}

\thusetup{
  cite-style = inline,
}

参见参考资料\citep{wangyouyuan}。

\section{侵蚀燃烧的计算}

\subsection{星型装药}

对于星型装药,我们采用æ准则拟合沿装药长度平均侵蚀比:
\[
    \overline{\varepsilon }=\overline{\varepsilon }\left( \varphi \right) =\left\{ \begin{matrix}
        1&		(\varphi \le 72.9)\\
        1.3128-1.3249\times 10^{-2}\varphi +1.5527\times 10^{-4}\varphi ^2-4.3868\times 10^{-7}\varphi ^3&		(\varphi >72.9)\\
    \end{matrix} \right. 
\]

式中:$\varphi$——燃通比,$\varphi=A_{b}/A{p}$

\qquad\quad$A_{b}$——燃面,$m^{2}$

\qquad\quad$A{p}$——通气面积,$m^{2}$

\subsection{多孔装药}

对于多孔装药,我们采用下式拟合沿装药长度平均侵蚀比:
\[
  \overline{\varepsilon }=\overline{\varepsilon }\bigl( p^{\alpha}\lambda ^{\beta} \bigr) =\left\{ \begin{matrix}
    1&		\bigl( p^{\alpha}\lambda ^{\beta}\le 0.495 \bigr)\\
    0.4578+0.2986\bigl( p^{\alpha}\lambda ^{\beta} \bigr) +1.6079\bigl( p^{\alpha}\lambda ^{\beta} \bigr) ^2&		\bigl( p^{\alpha}\lambda ^{\beta}>0.495 \bigr)\\
  \end{matrix} \right. 
\]

式中:$p$——装药末端压力

\qquad\quad$\lambda $——装药末端速度系数