% !TeX root = ../thuthesis-example.tex

\chapter{装药设计相关知识}

\section{推进剂选择原则}

推进剂应具有所需要的能量特性以比冲与密度的乘积来表示的。
\[
    I=I_{sp}m_pg=I_{sp}\rho _gg\mathrm{V}_p
\]
$V_{p}$一定时,$I_{sp}\rho _g$越大,$I$越大。相反$I$一定时,$I_{sp}\rho _g$越大,$V_{p}$越小,燃烧室体积也越小,燃烧室壳体质量也就越小。$I_{sp}$越大,$m_{p}$越小,发动机总质量也越小。所以,我们要求推进剂应该具有尽量高的能量特性。

\section{药形种类}

目前在发动机中采用的装药有很多,可以按照燃烧面积的变化规律、燃烧表面所处的位置划分,按侧面燃耗药柱划分,按燃烧方向划分。

按照燃烧面积的变化规律划分:
\begin{itemize}[leftmargin=2em]
    \item 恒面性药柱
    \item 减面性药柱
    \item 增面性药柱
\end{itemize}

按照燃烧表面所处的位置划分:
\begin{itemize}[leftmargin=2em]
    \item 端面燃烧药柱
    \item 侧面燃烧药柱
    \item 侧端面同时燃烧药柱
\end{itemize}

按照侧面燃耗药柱划分:
\begin{itemize}[leftmargin=2em]
    \item 外燃药柱
    \item 内燃药柱
    \item 内外燃药柱
\end{itemize}

按照燃烧方向划分:
\begin{itemize}[leftmargin=2em]
    \item 一维药柱
    \item 二维药柱
    \item 三维药柱
\end{itemize}

\subsection{一维药柱}

一维药柱即端燃药柱,这种药柱的侧表面及其一端是用包覆层阻燃的。燃烧只在一端进行,燃烧方向垂直于端面,因此是恒面性燃烧的药柱。

一维药柱具有恒面燃烧、工作时间长、装填系数最大、无压力峰、形状简单、制造容易、强度高等优点。但同时也具有推理小、重心偏移大、点火困难等不可避免的缺点。

\subsection{二维药柱}

二维药柱即侧燃药柱,由于侧燃药柱燃烧面积大,能够产生足够大的推力,在固体火箭发动机中被广泛采用。二维药柱种类很多,每种药柱都具有特定的燃烧面变化规律。

\begin{longtable}{c l l}
    \caption{不同二维药柱优缺点比较}
    \label{tab:longtable} \\
    \toprule
    种类 & \multicolumn{1}{c}{优点} & \multicolumn{1}{c}{缺点}  \\
    \midrule
  \endfirsthead
    \caption*{续表~\thetable\quad 不同二维药柱优缺点比较} \\
    \toprule
    种类 & \multicolumn{1}{c}{优点} & \multicolumn{1}{c}{缺点}  \\
    \midrule
  \endhead
    \bottomrule
  \endfoot
  
  \multirow{4}*{管型药柱} & 燃面恒定 & ~\\
 

  ~ & 结构简单,易于制造 & 药柱需要支撑装置 \\



  ~ & 无应力集中,强度大 &不可长时间工作 \\


  ~ & 无剩药 & \\

  \hline

  管套形药柱 & \begin{tabular}[c]{@{}l@{}}燃面恒定\\无剩药\end{tabular} & 支撑装置的工作条件恶劣\\
  \hline
  星形药柱 & \begin{tabular}[c]{@{}l@{}}可获得恒面性、减面性或增面性的燃烧\\工作时间可以很长\\可直接浇注在燃烧室内\\对壳体的刚度有增强作用\end{tabular} & \begin{tabular}[c]{@{}l@{}}药形复杂,药模制造困难\\药柱强度低,易出现裂纹\\有剩药\\推力、压力曲线\\有较长的拖尾现象\end{tabular}  \\
  \hline
  车轮形药柱 & \begin{tabular}[c]{@{}l@{}}可获得恒面性、减面性或增面性的燃烧\\工作时间可以很长\\肉厚系数低\\可以得到多推力方案\end{tabular}                                     & \begin{tabular}[c]{@{}l@{}}药柱强度低\\装填系数低,形状更为复杂\end{tabular}                              \\
  \hline
  多孔药柱  & 药柱绝热,可采用直接浇注工艺&\\
\end{longtable}

  