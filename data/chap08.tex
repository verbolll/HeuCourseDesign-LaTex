% !TeX root = ../thuthesis-example.tex

\chapter{内弹道的分析}

\section{分析}

压强时间曲线的形状和燃烧面积时间变化曲线的形状吻合较好。

在燃烧室压强上升段,侵蚀燃烧使得压强上升加快。侵蚀燃烧情况下,初始压强峰不大,与初始增面阶段的终点时刻的压强相比,初始压强峰的数值小,所以对燃烧室结构强度有利,能选择更薄的壳体,减轻发动机结构质量,提高重量比冲。

无论是否发生侵蚀燃烧,p-t图都是5号和7号接近,其他五组较为接近。无论是否发生侵蚀燃烧,1、2、3、4、6号p-t曲线重合情况比5、7号曲线更好。