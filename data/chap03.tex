% !TeX root = ../thuthesis-example.tex

\chapter{固体火箭发动机的总体设计}

\section{发动机结构形式}

\subsection{发动机结构形式的分类}

\begin{enumerate}[leftmargin=0em,itemindent=2em]
    \item 按药柱类型
    
    \qquad 根据药柱的类型,可以分为端燃药柱、内燃药柱和内外燃药柱。端燃药柱的燃烧面积较小,适合小推力和长时间运行的发动机;内燃药柱则能在内孔表面全长上同时燃烧,燃烧面积大,适用于大推力和长时间工作的情况;内外燃药柱可以在内外表面同时燃烧,满足大推力和短时间工作的需求。

    \item 按药柱装填方式
    
    \qquad 根据药柱的装填方式,发动机可以分为自由装填式和铸装式。自由装填式药柱通常通过压伸成型或模具浇注的方式制成,然后装入燃烧室。这种药柱具有较高的承压强度、良好的生产经济性和较好的储存安全性。而铸装式药柱则是将推进剂直接浇注到燃烧室中,并经过固化形成。这种方式不受工艺条件的限制,燃烧室壳体也不会与高温气体直接接触,适合用于大推力和长时间工作的发动机。

    \item 按喷管数目分
    
    \qquad 根据喷管的数量,发动机可以分为单喷管式和多喷管式。通常情况下,许多发动机采用单喷管设计,因为这种设计相对简单,易于制造和维护。在某些特殊情况下,比如发动机的长度受到限制,或者需要通过喷管来实现旋转和滚转控制时,会采用多喷管设计。多喷管式发动机通常能够提供更灵活的推力控制,使得航天器在飞行过程中能够进行更精确的姿态调整和轨道变换。这种设计可以通过多个喷管同时或独立工作,产生不同的推力矢量,从而有效地控制飞行器的运动状态。此外,多喷管系统还可以在故障发生时提供冗余,增强发动机的可靠性。

    \item 按喷管形式
    
    \qquad 
    根据喷管的设计,发动机可以分为普通喷管式和潜入喷管式。普通喷管式是大多数发动机的标准设计,适用于大多数常规应用,其喷管外形通常较为简单,能够有效地引导气流并优化推力。而潜入喷管式设计则适用于空间有限的情况。在这种设计中,喷管通常被置于机身内部或与机身紧密结合,以减少外部干扰和提高气动性能。潜入喷管可以有效降低噪音、减少气流分离现象,并且在特定情况下提升发动机的整体效率。

    \item 按推力级数
    
    \qquad 根据推力级数,发动机可以分为单推力和双推力两种类型。单推力发动机只有一个推力阶段,适用于许多传统应用,如常规火箭和一些类型的导弹。双推力发动机则具有两个推力阶段,其中第一级提供大推力以支持起飞,而第二级则提供较小的推力以维持飞行,通常用于小型战术导弹。双推力发动机又可以细分为单室双推力和双室双推力,其中双室双推力又分为串联式和并联式。
\end{enumerate}

\subsection{发动机结构形式的选择原则}

在选择发动机结构形式时应该遵循以下原则:
\begin{enumerate}[leftmargin=2em]
    \item 能适应发动机要求的用途和战术技术性能
    \item 设计的发动机重量轻、结构紧凑
    \item 使发动机具有良好的工艺性、经济性、研制周期短
\end{enumerate}

\subsection{根据结构形式的选择原则确定结构形式}

根据设计任务书的要求,考虑到发动机的工作时间较长,决定采用内燃药柱。为了方便药柱成型,选择了铸装式结构,能够直接将推进剂浇注到燃烧室中,从而确保药柱的形状不受工艺条件的限制。

\section{发动机壳体材料}

\subsection{常见壳体材料}

目前用于固体火箭发动机的壳体材料主要分为金属材料和非金属材料两类。

\noindent 金属材料
\begin{enumerate}[leftmargin=2em]
    \item 优质碳素结构钢
    
    具有价格低、来源丰富和良好的工艺性,适合大规模生产。然而,其强度和耐热性较差,通常用于小直径的野战火箭发动机壳体。

    \item 一般合金钢
    
    在冷热加工和焊接方面表现良好,能够承受短时间的高温,成型性能也不错。但它存在回火脆性,因此需要在高温回火后缓慢冷却。广泛应用于中小型战术导弹发动机的壳体。

    \item 超高强度合金钢
    
    具有高比强度,能显著减轻结构重量,但其缺口敏感性较高。在焊接和热处理时需要严格控制工艺并进行仔细检验,主要用于大型固体火箭发动机的壳体。

    \item 高强度铝合金
    
    比强度高且刚性良好,但耐热性和焊接性能较差,缺口敏感性较大。一般只在具有绝热内衬、工作时间短的内燃药柱或壳体中使用。

    \item 钛合金
    
    具有极高的比强度,耐高温和腐蚀,旋压加工性能良好,冲击韧性优于合金钢。缺点是切削和焊接性能较差,弹性模量低,结构刚性较弱,成本较高,主要应用于航天发动机。

\end{enumerate}

\noindent 非金属材料

复合材料

\hangindent=2em  复合材料具有高比强度,缠绕工艺简单,易于实现机械化和自动化,尺寸不受限制,可以整体成型,抗振性和绝热性较好。但其纤维强度较低,壁厚较厚,工艺质量不够稳定,且长期存储会出现老化现象。复合材料被认为是固体火箭发动机未来发展的有前景的壳体材料。

\subsection{壳体材料选择的要求}

选择发动机壳体材料时应满足以下要求:

\begin{enumerate}[leftmargin=2em]
    \item 比强度高,满足发动机结构质量的要求
    \item 韧性好,保证壳体不发生脆性破坏
    \item 刚度高,防止壳体发生失稳
    \item 工艺性好,便于成形和加工
    \item 经济性好,降低发动机的生产制造成本
\end{enumerate}

\section{发动机推进剂}

\subsection{固体推进剂的分类}

固体推进剂主要分为双基推进剂、复合推进剂和复合双基推进剂三类。

\begin{enumerate}[leftmargin=2em]
    \item 双基推进剂的主要成分包括硝化纤维素和硝化甘油。为了提升其性能,常添加增塑剂、化学稳定剂和铅盐等成分。该推进剂具有较高的机械强度、较低的火焰温度,以及良好的储存安全性,且对潮湿环境不敏感,便于大规模生产,并且重复性良好。

    \item 复合推进剂的基本成分由氧化剂、粘合剂和金属燃料组成。通常还会添加一些其他成分,例如固化剂、防老剂、增塑剂和燃烧催化剂,以提升推进剂的性能。其比冲高,燃烧速率可调范围广,压强指数和温度系数较低,燃速对温度变化的敏感性较小,并且在低温条件下表现出良好的机械性能。

    \item 复合双基推进剂(改性双基药)是在双基推进剂中加入过氯酸铵和金属粉末,以增强其能量特性。该推进剂展现出较高的能量特性、比冲和燃烧效率,同时具有较高的燃烧速率,但在低温下的机械性能相对较差。
\end{enumerate}

\subsection{推进剂选择的要求}

选择发动机推进剂时应遵循以下要求:

\begin{enumerate}[leftmargin=2em]
    \item 满足需要的能量特性
    \item 满足要求的内弹道特性
    \item 良好的燃烧特性
    \item 足够的力学特性
    \item 良好的物理、化学安定性
    \item 危险性小
    \item 生产经济性好
    \item 在选择推进剂时还可能提出无烟、使无线电波衰减小、易于点燃等要求
\end{enumerate}