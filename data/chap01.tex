% !TeX root = ../thuthesis-example.tex
\chapter{绪论}
固体火箭发动机是火箭和导弹中广泛应用的动力装置,为了实现我国国防现代化建设大业,作为相关专业的学生应该具备设计一款性能符合要求结构安全可靠的固体火箭发动机的能力。

固体火箭发动机由于具有结构简单、体积小、工作可靠、操作简便、能够长期贮存等优点,即使在今天仍然没有被淘汰依旧具有很大的研究学习价值。
\section{固体火箭发动机的结构}
固体火箭发动机主要由燃烧室、装药、点火装置和喷管等部件组成。
\begin{enumerate}[leftmargin=2em]
  \item 燃烧室:
  
  发动机中的关键装置,提供推进剂燃烧空间,将化学能转化为高温高压的热能,并采用高性能金属或复合材料作为壳体承受高温高压,同时周围涂有绝热层采取隔热措施保护壳体。
  \item 装药:
  
  固体推进剂制成的药柱,作为发动机的能源和工质源,需具备特定几何形状,并通过控制燃烧表面变化,以实现预期推力。包括\textbf{双基(double base,DB)推进剂}和\textbf{复合推进剂(compositepropellant)}。

  \item 点火装置:
  
  由保险机构和点火器组成,用于点燃发动机装药,使发动机顺利起动。点火器中有接受起动信息就开始工作的始发器如电发火管,还有相当数量的点火药。对于大型的固体发动机,其点火器往往设计成了小型的点火发动机。

  \item 喷管:
  
  发动机中关键的能量转换装置,通过加速膨胀高温高压燃气以产生推力,同时也并具备推力向量控制和推力终止装置来实现飞行器的姿态控制和精确停机。
\end{enumerate}
% \section{固体火箭发动机设计总要求}


% 火箭发动机课程设计是飞行器动力工程专业的必修实践课,旨在让学生掌握固体发动机设计的基本方法和原理。课程任务包括:根据给定设计要求选择药形,进行装药设计,计算发动机内弹道性能,分析装药特点,并筛选出最佳方案。

% 选择一种药形,采用附录A中表1-3所提供的推进剂,设计一组装药,进行发动机内弹道计算,通过内弹道分析和装药性能对比,最终确定一组装药。题目为“xx形装药设计”。

% \section{固体火箭发动机设计详细参数要求}

% \subsection{装药几何参数}

% \begin{table}
%   \centering
%   \caption{装药几何参数}
%   \begin{tabular}{ll}
%     \toprule
%     相关几何参数          & 要求                         \\
%     \midrule
%     外径   & 230~300mm \\
%     长度   & 1300~1800mm                     \\
%     总冲   & 不小于230kN·s    \\
%     最大推力   & 38~72kN    \\
%     平均推力   & 35~65kN    \\
%     最小推力   & 31~58kN    \\
%     平均工作时间   & 4~7s    \\
%     工作环境温度   & 20℃    \\

%     \bottomrule
%   \end{tabular}
%   \label{tab:three-line}
% \end{table}

% \subsection{进剂种类选择}

% 在附录A表1~3中选取.

% \subsection{壳体材料选择}

% \thusetup{
%   cite-style = inline,
% }

% 参见参考资料\citep{wangyouyuan}。
\section{固体火箭发动机设计原始参数}
\begin{enumerate}[leftmargin=2em]
  \item 发动机的用途
  
  发动机用在特种火箭或导弹上。
  \item 发动机的总冲
  
  总冲量$I$用发动机推力$F$对时间变量$t$在整个发动机工作时间$t_{a}$区间内的积分来表示,即
  \[
    I=\int_{0}^{t_{a}} Fdt
  \]
  \item 发动机的比冲
  
  单位重量推进剂产生的冲量,即
  \[
    I_{sp}=\frac{I}{m_{p}g} 
  \]
  \item 发动机的使用温度范围
  
  由设计要求部门提供。

  \item 发动机的平均推力
  
  会给出在常温下的平均推力$t_{a}$,有时也会给出所限制的最长工作时间$t_{amax}$与最短工作时间$t_{amin}$。

  \item 发动机的工作时间
  
  会给出在常温下的工作时间$\bar{F} $,有时也会给出所限制的最大推力$F_{max}$与最小推力$F_{min}$。

  \item 发动机的推力方案
  
  发动机选择等推力还是变推力。

  \item 发动机的质量限制
  
  限制发动机的总质量$m$和结构质量$m_{m}$。

  \item 发动机的尺寸限制
  
  对发动机燃烧室直径$D$、长径比$L/D$的限制。

  \item 发动机性能的偏差量
  
  包括推力、工作时间和总冲的偏差量,和允许的推力偏角等。

  \item 点火延迟期
  
  点火延迟期$t_{ig}$是以推力(或压力)达到其额定值的75$\sim$80\%所需的时间给出的。

  \item 对推力矢量控制装置的要求
  
  如有无推力矢量控制装置,推力矢量控制装置的最大侧向力最大偏转角的大小和频率响应等。

  \item 对推力终止装置的要求
  
  如有无推力终止装置,推力终止的时间和作用效果等。

  \item 发动机的贮存期限
  
  有关部门提出贮存要求。

  \item 发动机的运输条件
  
  包括运输方式、运输距离和行车速度等。

  \item 特殊要求
  

  
\end{enumerate}
\section{固体火箭发动机设计任务要求}

固体火箭发动机设计内容包括:

\begin{itemize}[leftmargin=2em]
  \item 发动机总体设计
  \item 发动机装药设计
  \item 发动机燃烧室设计
  \item 发动机喷管设计
  \item 发动机点火装置设计
  \item 推力矢量控制装置设计
  \item 推力终止装置设计
\end{itemize}

固体火箭发动机设计需要考虑的因素

\begin{itemize}[leftmargin=2em]
  \item 良好的可靠性
  \item 良好的安全性
  \item 经济型
\end{itemize}
