% !TeX root = ../thuthesis-example.tex

\begin{denotation}[3cm]
  \item[$A_{c}$] 装药横截面积
  \item[$A_{f}$] 剩药横截面积
  \item[$A_{p}$] 燃烧室通气面积
  \item[$A_{t}$] 喷管喉部面积
  \item[$C_{+20^\circ C}^*$] 特征速度
  \item[$C_{F+20^\circ C}$] 推力系数
  \item[$D$] 药柱外径
  \item[$\overline{F}_{+20^\circ C}$] 20℃平均推力
  \item[$I$] 总冲量
  \item[$I_{sp}$] 理论比冲量
  \item[$J$] 通气参量
  \item[$\overline{K_N}$] 面喉比
  \item[$\overline{P}_{+20^\circ C}$] 燃烧室平均压强
  \item[$\overline{S}$] 燃面平均值
  \item[$a$] 燃速系数
  \item[$e_{1}$] 总燃层厚
  \item[$L$] 药柱长度
  \item[$k$] 绝热指数
  \item[$l$] 药柱特征长度
  \item[$m_{f}$] 剩药质量
  \item[$m_{p}$] 装药质量
  \item[$m_{peff}$] 有效装药质量
  \item[$n$] 燃速压力指数
  \item[$p_{a}$] 环境压力
  \item[$p_{c}$] 燃烧室压力
  \item[$p_{e}$] 喷管出口压力
  \item[$r$] 过渡圆弧半径
  \item[$r_{1}$] 星角圆弧半径
  \item[$r_{\fontsize{8}{14}\mbox{燃}}$] 推进剂燃速
  \item[$y$] 相对燃层厚
  \item[$\varGamma$] 比热比函数
  \item[$\varepsilon$] 角度系数
  \item[$\varepsilon_{\fontsize{8}{14}\selectfont\mbox{侵}}$] 侵蚀比
  \item[$\eta$] 装填系数
  \item[$\eta_{f}$] 剩药系数
  \item[$\theta/2$] 星边半角
  \item[$\rho_{P}$] 推进剂密度
  \item[$\varphi$] 燃通比
\end{denotation}



% 也可以使用 nomencl 宏包,需要在导言区
% \usepackage{nomencl}
% \makenomenclature

% 在这里输出符号说明
% \printnomenclature[3cm]

% 在正文中的任意为都可以标题
% \nomenclature{PI}{聚酰亚胺}
% \nomenclature{MPI}{聚酰亚胺模型化合物,N-苯基邻苯酰亚胺}
% \nomenclature{PBI}{聚苯并咪唑}
% \nomenclature{MPBI}{聚苯并咪唑模型化合物,N-苯基苯并咪唑}
% \nomenclature{PY}{聚吡咙}
% \nomenclature{PMDA-BDA}{均苯四酸二酐与联苯四胺合成的聚吡咙薄膜}
% \nomenclature{MPY}{聚吡咙模型化合物}
% \nomenclature{As-PPT}{聚苯基不对称三嗪}
% \nomenclature{MAsPPT}{聚苯基不对称三嗪单模型化合物,3,5,6-三苯基-1,2,4-三嗪}
% \nomenclature{DMAsPPT}{聚苯基不对称三嗪双模型化合物(水解实验模型化合物)}
% \nomenclature{S-PPT}{聚苯基对称三嗪}
% \nomenclature{MSPPT}{聚苯基对称三嗪模型化合物,2,4,6-三苯基-1,3,5-三嗪}
% \nomenclature{PPQ}{聚苯基喹噁啉}
% \nomenclature{MPPQ}{聚苯基喹噁啉模型化合物,3,4-二苯基苯并二嗪}
% \nomenclature{HMPI}{聚酰亚胺模型化合物的质子化产物}
% \nomenclature{HMPY}{聚吡咙模型化合物的质子化产物}
% \nomenclature{HMPBI}{聚苯并咪唑模型化合物的质子化产物}
% \nomenclature{HMAsPPT}{聚苯基不对称三嗪模型化合物的质子化产物}
% \nomenclature{HMSPPT}{聚苯基对称三嗪模型化合物的质子化产物}
% \nomenclature{HMPPQ}{聚苯基喹噁啉模型化合物的质子化产物}
% \nomenclature{PDT}{热分解温度}
% \nomenclature{HPLC}{高效液相色谱(High Performance Liquid Chromatography)}
% \nomenclature{HPCE}{高效毛细管电泳色谱(High Performance Capillary lectrophoresis)}
% \nomenclature{LC-MS}{液相色谱-质谱联用(Liquid chromatography-Mass Spectrum)}
% \nomenclature{TIC}{总离子浓度(Total Ion Content)}
% \nomenclature{\textit{ab initio}}{基于第一原理的量子化学计算方法,常称从头算法}
% \nomenclature{DFT}{密度泛函理论(Density Functional Theory)}
% \nomenclature{$E_a$}{化学反应的活化能(Activation Energy)}
% \nomenclature{ZPE}{零点振动能(Zero Vibration Energy)}
% \nomenclature{PES}{势能面(Potential Energy Surface)}
% \nomenclature{TS}{过渡态(Transition State)}
% \nomenclature{TST}{过渡态理论(Transition State Theory)}
% \nomenclature{$\increment G^\neq$}{活化自由能(Activation Free Energy)}
% \nomenclature{$\kappa$}{传输系数(Transmission Coefficient)}
% \nomenclature{IRC}{内禀反应坐标(Intrinsic Reaction Coordinates)}
% \nomenclature{$\nu_i$}{虚频(Imaginary Frequency)}
% \nomenclature{ONIOM}{分层算法(Our own N-layered Integrated molecular Orbital and molecular Mechanics)}
% \nomenclature{SCF}{自洽场(Self-Consistent Field)}
% \nomenclature{SCRF}{自洽反应场(Self-Consistent Reaction Field)}
